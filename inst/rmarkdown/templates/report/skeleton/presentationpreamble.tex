\usepackage{appendixnumberbeamer}

\usepackage{booktabs}
\usepackage[scale=2]{ccicons}

\usepackage{pgfplots}
\usepgfplotslibrary{dateplot}

\usepackage{xspace}
%\newcommand{\themename}{\textbf{\textsc{metropolis}}\xspace}

\usepackage{enumerate}

\usepackage{OAKdefn}

\usepackage{amssymb, amsmath, amsfonts, latexsym, mathtools, nccmath, xfrac}
\usepackage{graphicx}
\usepackage{optidef}
\usepackage{bbm}

%%%%%%%%%%%%%%%%%%%%%%%%%%%%%%%%%%%%%%%%
% 	Colors
% 	inspired by https://paulgp.github.io/beamer_tips.pdf
%%%%%%%%%%%%%%%%%%%%%%%%%%%%%%%%%%%%%%%%

\definecolor{blue}{RGB}{0,114,178}
\definecolor{red}{RGB}{128,0,0}
\definecolor{yellow}{RGB}{240,228,66}
\definecolor{green}{RGB}{0,158,115}
\definecolor{Purple}{HTML}{911146}


% Signature Color Palette
	\definecolor{ChicagoMaroon}{RGB}{128,0,0}
	\definecolor{ChicagoDarkGray}{RGB}{118,118,118}
	\definecolor{ChicagoLightGray}{RGB}{214,214,206}
	\definecolor{MainText}{RGB}{0, 0, 0}

% Secondary Color Palette
	\definecolor{ChicagoRed}{RGB}{143,57,49}
	\definecolor{ChicagoOrange}{RGB}{193,102,34}
	\definecolor{ChicagoLightGreen}{RGB}{138,157,69}
	\definecolor{ChicagoDarkGreen}{RGB}{101,109,51}
	\definecolor{ChicagoBlue}{RGB}{21,95,131}
	\definecolor{ChicagoBlueLight}{RGB}{91,150,173}
	\definecolor{ChicagoBlueDark}{RGB}{21,67,95}
	\definecolor{ChicagoViolet}{RGB}{53,14,32}

% Colors for the Shading
	\definecolor{RedShade1}{RGB}{138,0,0} % Colors taken from UChicago Powerpoint slides
	\definecolor{RedShade2}{RGB}{68,0,0}



\definecolor{MyBackground}{RGB}{250, 250, 250}

\setbeamercolor{background canvas}{bg=MyBackground}
\definecolor{StandoutBackground}{RGB}{202, 215, 227}

\newenvironment{transitionframe}{
\setbeamercolor{background canvas}{bg=StandoutBackground}
\begin{frame}}{
\end{frame}
}

\setbeamercolor{frametitle}{fg=white, bg = red}
\setbeamercolor{title}{fg=black}
\setbeamertemplate{footline}[frame number]
\setbeamertemplate{navigation symbols}{}
\setbeamercolor{itemize item}{fg=ChicagoMaroon}
\setbeamercolor{itemize subitem}{fg=ChicagoMaroon}
\setbeamercolor{enumerate item}{fg=ChicagoMaroon}
\setbeamercolor{enumerate subitem}{fg=ChicagoMaroon}
\setbeamercolor{button}{bg=ChicagoMaroon, fg=MyBackground}




\setbeamercolor{block title alerted}{fg=white,bg=ChicagoBlue}
\setbeamercolor{block body alerted}{bg=ChicagoBlueLight!50!white}

\setbeamercolor{block title}{bg=ChicagoMaroon,fg=white}
\setbeamercolor{block body}{bg=ChicagoLightGray}

\setbeamercolor{block title example}{bg=ChicagoDarkGreen,fg=white}
\setbeamercolor{block body example}{bg=ChicagoLightGreen!25!white}


\setbeamercolor*{enumerate item}{fg=ChicagoMaroon}
\setbeamercolor*{enumerate subitem}{fg=ChicagoMaroon}
\setbeamercolor*{enumerate subsubitem}{fg=ChicagoMaroon}


\setbeamercolor*{itemize item}{fg=ChicagoMaroon}
\setbeamercolor*{itemize subitem}{fg=ChicagoMaroon}
\setbeamercolor*{itemize subsubitem}{fg=ChicagoMaroon}




%%%%%%%%%%%%%%%%%%%%%%%%%%%%%%%%%%%%%%%%
% 	ToC at beginning of every section
%%%%%%%%%%%%%%%%%%%%%%%%%%%%%%%%%%%%%%%%
\defbeamertemplate{section in toc}{sections numbered roman}{%
  \leavevmode%
  \MakeUppercase{\romannumeral\inserttocsectionnumber}.\ %
  \inserttocsection\par}

\AtBeginSection[]
  {
     \begin{frame}<beamer>
     \setbeamertemplate{section in toc}[sections numbered roman]
     \frametitle{Outline}
     \tableofcontents[currentsection]
     \end{frame}
  }

%%%%%%%%%%%%%%%%%%%%%%%%%%%%%%%%%%%%%%%%
% 	Blocks
%%%%%%%%%%%%%%%%%%%%%%%%%%%%%%%%%%%%%%%%

\newenvironment<>{definition}[1]{%
  \setbeamercolor{block title}{fg=black,bg=lightgray}
  \begin{block}#2{#1}}{\end{block}}
\setbeamertemplate{blocks}[rounded][shadow=false]



%%%%%%%%%%%%%%%%%%%%%%%%%%%%%%%%%%%%%%%%
% 	Footer
%%%%%%%%%%%%%%%%%%%%%%%%%%%%%%%%%%%%%%%%

\setbeamertemplate{footline}{%
  \begin{beamercolorbox}[wd=\textwidth, sep=3ex]{footline}%
    \usebeamerfont{page number in head/foot}%
    \usebeamertemplate*{frame footer}
    \hfill%
    \insertsection \quad
    \insertframenumber/\inserttotalframenumber
  \end{beamercolorbox}%
  }

%%%%%%%%%%%%%%%%%%%%%%%%%%%%%%%%%%%%%%%%
% 	Formatting
%%%%%%%%%%%%%%%%%%%%%%%%%%%%%%%%%%%%%%%%

%\setbeameroption{show notes} % un-comment to see the notes
\newcommand*{\myparskip}{\vspace{2em}}


\setbeamertemplate{frametitle continuation}[from second][\insertcontinuationcountroman]

\newenvironment{wideitemize}{\setlength{\leftmargini}{0pt}\itemize\addtolength{\itemsep}{1em}}{\enditemize}

\DeclareMathOperator{\argmin}{argmin}


